%% 
%% Copyright 2019-2020 Elsevier Ltd
%% 
%% This file is part of the 'CAS Bundle'.
%% --------------------------------------
%% 
%% It may be distributed under the conditions of the LaTeX Project Public
%% License, either version 1.2 of this license or (at your option) any
%% later version.  The latest version of this license is in
%%    http://www.latex-project.org/lppl.txt
%% and version 1.2 or later is part of all distributions of LaTeX
%% version 1999/12/01 or later.
%% 
%% The list of all files belonging to the 'CAS Bundle' is
%% given in the file `manifest.txt'.
%% 
%% Template article for cas-dc documentclass for 
%% double column output.

%\documentclass[a4paper,fleqn,longmktitle]{cas-dc}
\documentclass[a4paper,fleqn]{cas-dc}

%\usepackage[authoryear,longnamesfirst]{natbib}
%\usepackage[authoryear]{natbib}
\usepackage[ruled, algonl, vlined]{algorithm2e}
\SetAlFnt{\small}
\SetAlCapFnt{\small}
\SetAlCapNameFnt{\small}
\usepackage{color}
\usepackage[numbers]{natbib}
\usepackage{amssymb}
\usepackage{amsmath}
\usepackage{amsthm} 
\usepackage[nocfg]{nomencl}
\usepackage{ifthen}
\usepackage{paralist}
\usepackage{tabularx}
\usepackage{graphics}
\usepackage{epsfig}
\usepackage{graphicx}
\usepackage{epstopdf}
\usepackage{multirow}
\usepackage{siunitx}
\usepackage{amsmath,amssymb}
\usepackage{graphicx}
\usepackage{float}

\usepackage{tabularx}
\newcolumntype{Y}{>{\raggedright\arraybackslash}X}

\usepackage{float}
\usepackage{subfigure}
\usepackage{booktabs}
\usepackage{array}
\usepackage{pifont}
\newcommand{\cmark}{\ding{51}}%
\newcommand{\xmark}{\ding{55}}%

%% Nomenclature
\usepackage{framed} % Framing content
\usepackage{multicol} % Multiple columns environment
\usepackage{nomencl} % Nomenclature package
\makenomenclature
\setlength{\nomitemsep}{-\parskip} % Baseline skip between items
\renewcommand*\nompreamble{\begin{multicols}{2}}
\renewcommand*\nompostamble{\end{multicols}}
\usepackage{etoolbox}

\newtheorem{proposition}{Proposition}
\newtheorem{remark}{Remark}
%%%Author definitions
\def\tsc#1{\csdef{#1}{\textsc{\lowercase{#1}}\xspace}}
\tsc{WGM}
\tsc{QE}
\tsc{EP}
\tsc{PMS}
\tsc{BEC}
\tsc{DE}
%%%

% ---------- Clean the CAS template front-matter artifacts ----------
\ExplSyntaxOn

% Remove running head title (kills "Short Title of the Article")
\cs_gset:Npn \__short_title: {}

% Remove footer author string and "Preprint submitted to Elsevier"
\cs_gset:Npn \__short_authors: {}
\cs_gset:Npn \__first_footerline: {}  % keeps "Page X of Y" but removes the preprint line

% Remove ORCID(s): line
\RenewDocumentCommand \printorcid {} {}

% Remove "A B S T R A C T" label
\tex_gdef:D \abstractname {}

% Remove the 2 full-width rules around the abstract box (top+bottom lines)
\RenewDocumentCommand \dashrule { O{.4pt} m m } { \skip_vertical:n {#2} \skip_vertical:n {#3} }

% Remove the small middle rule under ABSTRACT by redefining the Abstract environment
\RenewDocumentEnvironment { Abstract } { o }
  { \group_begin: \footnotesize \ignorespaces }
  { \par \group_end: }

\ExplSyntaxOff
% -------------------------------------------------------------------

\begin{document}
\let\WriteBookmarks\relax
\def\floatpagepagefraction{1}
\def\textpagefraction{.001}

%%\tnotemark[1,2]

%%\tnotetext[1]{Note 1}

%%\tnotetext[2]{Note 2}


% Here goes the abstract


% Use if graphical abstract is present
%\begin{graphicalabstract}
%\includegraphics{}
%\end{graphicalabstract}

% Research highlights
% Research highlights

% Keywords
% Each keyword is seperated by \sep
\maketitle
\scriptsize

\begin{table}[t]
\scriptsize
\centering
$\vspace{-10mm}$
\caption{Variables of the climate dynamic model.}
\begin{tabularx}{\linewidth}{@{}lYl@{}}
\toprule
Notation               & Definition                     & Unit \\
\midrule
$C^{\mathrm{in}}$      & Indoor CO$_2$ absolute concentration.            & $\mathrm{g/m^{3}}$ \\
$C^{\mathrm{out}}$     & Outdoor CO$_2$ absolute concentration.           & $\mathrm{g/m^{3}}$ \\

$D^{\mathrm{dos}}$   & CO$_2$ dosing rate.                      & $\mathrm{g/s}$\\
$D^{\mathrm{vent}}$  & CO$_2$ changing rate due to ventilation.  & $\mathrm{g/s}$\\
$D^{\mathrm{ass}}$   & CO$_2$ assimilation rate.                & $\mathrm{g/s}$\\

$F^{\mathrm{vent}}$    & Humidity changing rate due to venting.      & $\mathrm{g/s}$ \\
$F^{\mathrm{cool}}$    & Humidity changing rate due to evaporative cooling and humidification.      & $\mathrm{g/s}$ \\
$F^{\mathrm{pad}}$     & Humidity addition rate due to evaporative cooling pads.      & $\mathrm{g/s}$ \\
$F^{\mathrm{dehum}}$   & Dehumidifying rate.                         & $\mathrm{g/s}$ \\
$F^{\mathrm{tran}}$    & Transpiration vapor rate.                   & $\mathrm{g/s}$ \\
$F^{\mathrm{vc}}$      & Condensation vapor exchange rate.           & $\mathrm{g/s}$ \\

$g^{\mathrm{vc}}$    & Cover–air mass transfer conductance.     & $\mathrm{m/s}$\\
$g^{\mathrm{tran}}$  & Transpiration conductance.               & $\mathrm{m/s}$\\

$H^{\mathrm{in}}$      & Indoor absolute humidity.                & $\mathrm{g/m^{3}}$\\
$H^{\mathrm{in,sat}}$  & Indoor saturation absolute humidity.     & $\mathrm{g/m^{3}}$\\
$H^{\mathrm{out}}$     & Outdoor absolute humidity.               & $\mathrm{g/m^{3}}$\\
$H^{\mathrm{cano}}$    & Absolute humidity at canopy.             & $\mathrm{g/m^{3}}$\\

$L^{\mathrm{DLI}}$     & Daily light integral (resets to 0 at 00:00).            & $\mathrm{mol/m^{2}}$\\

$Q^{\mathrm{heat}}$  & Heating power.                           & $\mathrm{W}$ \\
$Q^{\mathrm{vent}}$  & Ventilation heat-exchange power.         & $\mathrm{W}$ \\
$Q^{\mathrm{cool}}$  & Cooling heat-removal power.              & $\mathrm{W}$ \\
$Q^{\mathrm{solar}}$ & Solar heat-gain power.                   & $\mathrm{W}$ \\
$Q^{\Delta T}$       & Conduction heat-loss power.              & $\mathrm{W}$ \\
$Q^{\mathrm{LED}}$   & LED heat-gain power.                     & $\mathrm{W}$ \\
$Q^{\mathrm{tran}}$  & Transpiration latent heat-loss power.    & $\mathrm{W}$ \\

$r^{\mathrm{s}}$         & Canopy stomatal resistance.              & $\mathrm{s/m}$ \\
$R^{\mathrm{cano-abs}}$  & Canopy-absorbed radiation.               & $\mathrm{W/m^{2}}$ \\
$R^{\mathrm{cano-glo}}$  & Global radiation at canopy.              & $\mathrm{W/m^{2}}$ \\
$R^{\mathrm{cano-solar}}$& Solar radiation at canopy.        & $\mathrm{W/m^{2}}$ \\
$R^{\mathrm{cano-LED}}$  & LED radiation at canopy.          & $\mathrm{W/m^{2}}$ \\
$R^{\mathrm{out}}$       & Outdoor radiation.                       & $\mathrm{W/m^{2}}$ \\

$t$                    & Time.                                  & $\mathrm{s}$ \\
$T^{\mathrm{out}}$     & Outdoor temperature.                   & $\si{\celsius}$ \\
$T^{\mathrm{in}}$      & Indoor temperature.                    & $\si{\celsius}$ \\
$T^{\mathrm{cover}}$   & Covering temperature.                  & $\si{\celsius}$ \\
$\Delta T$             & Temperature difference.                & $\si{\celsius}$ \\ 

$U^{\mathrm{heat}}$       & Heater actuation level.                   & - \\
$U^{\mathrm{fan}}$        & Ventilation fan speed level.              & - \\
$U^{\mathrm{nat}}$        & Natural ventilation opening ratio.        & - \\
$U^{\mathrm{pad}}$        & Evaporative cooling pad wetness level.    & - \\
$U^{\mathrm{dos}}$        & CO$_2$ dosing rate command.               & - \\
$U^{\mathrm{LED}}$        & LED electric power command.               & - \\
$U^{\mathrm{scre-shad}}$  & Shading screen closure fraction.          & - \\
$U^{\mathrm{scre-warm}}$  & Warm-keep screen closure ratio.           & - \\
$U^{\mathrm{hum}}$        & Humidification rate command.              & - \\
$U^{\mathrm{dehum}}$      & Dehumidification rate command.            & - \\
$V^{\mathrm{vent}}$       & Total ventilation.            & $\mathrm{m^{3}}$ \\
$X^{\mathrm{cano}}$ & PPFD at canopy.                 & $\mathrm{\mu mol/(m^{2}\cdot s)}$  \\
\bottomrule       
\end{tabularx}
\end{table}

\begin{table}[t]
\scriptsize
\centering
$\vspace{-10mm}$
\caption{Parameters of the state-space model.}
\begin{tabularx}{\linewidth}{@{}lYll@{}}
\toprule
Notation               & Definition     & Value  & Unit \\
\midrule
$A^{\mathrm{floor}}$    & Floor area.      & 421    & $\mathrm{m^{2}}$ \\
$A^{\mathrm{roof}}$     & Roof area.       & 472    & $\mathrm{m^{2}}$ \\
$A^{\mathrm{wall}}$     & Side-wall area.  & 628    & $\mathrm{m^{2}}$ \\

$c^{\mathrm{vh}}$          & Air volumetric heat capacity.       & 1230 & $\mathrm{J/(m^{3} \cdot \si{\celsius})}$ \\
$\hat{C}^{\mathrm{in}}$    & CO$_2$ half-saturation constant. & 0.23         & $\mathrm{g/m^{3}}$ \\

$\bar{D}^{\mathrm{dos}}$   & Maximum CO$_2$ dosing rate.       & 1.04          & $\mathrm{g/s}$\\
$\bar{D}^{\mathrm{ass}}$   & Maximum CO$_2$ assimilation flux. & 2.2e-3        & $\mathrm{g/(m^{2}\cdot s)}$\\

$\bar{F}^{\mathrm{hum}}$    & Maximum injection rate of mist water.     & 18            & $\mathrm{g/s}$ \\
$\bar{F}^{\mathrm{dehum}}$  & Maximum moisture removal rate.            & 12            & $\mathrm{g/s}$ \\

$\tilde{H}$                      & Empirical coefficient.                  & 5.5638     & $\mathrm{g/m^{3}}$ \\

$L$                              & Leaf area index.                        & 2.5        & $\mathrm{m^{2}}/\mathrm{m^{2}}$ \\

$\rho$                     & PPFD response coefficient.         & 1.5e-3  & $\mathrm{(m^{2}\cdot s)/\mu mol}$ \\
$\bar{P}^{\mathrm{LED}}$   & Maximum electrical power of LEDs.  & 4.375e4   & $\mathrm{W}$ \\

$q^{\mathrm{evap}}$        & Latent heat of evaporation.        & 2430 &$\mathrm{J/g}$\\
$\bar{Q}^{\mathrm{heat}}$  & Maximum power from heaters.        & 2.36e5       & $\mathrm{W}$ \\

$r^{\mathrm{b}}$                 & Boundary layer resistance.              & 200     & $\mathrm{s/m}$ \\
$\bar{r}^{\mathrm{s}}$           & Maximum additional resistance.          & 570        & $\mathrm{s/m}$ \\
$\underline{r}^{\mathrm{s}}$     & Minimal stomatal resistance.            & 82         & $\mathrm{s/m}$ \\

$\upsilon^{\mathrm{roof}}$ & Heat-transfer coefficient of roofs.       & 6.6      & $\mathrm{W/(m^{2}\cdot \si{\celsius})}$ \\
$\upsilon^{\mathrm{wall}}$ & Heat-transfer coefficient of side walls.  & 6.3      & $\mathrm{W/(m^{2}\cdot \si{\celsius})}$ \\
$\nu^{\mathrm{surf}}$      & Condensation surface coefficient.         & 1.8e-3  & $\mathrm{m/(\si{\celsius}^{1/3}\cdot s)}$\\

$V^{\mathrm{gh}}$          & Indoor volume.                             & 3.351e3         & $\mathrm{m^{3}}$ \\
$\bar{V}^{\mathrm{fan}}$   & Maximum mechanical ventilation rate.       & 48            & $\mathrm{m^{3}/s}$ \\
$\bar{V}^{\mathrm{nat}}$   & Maximum natural ventilation rate.          & 5             & $\mathrm{m^{3}/s}$ \\

$\tau$                     & Crop-specific coefficient.                 & 0.4        & $\mathrm{m^{2}/W}$\\
$\Delta T^{\mathrm{pad}}$  & Air temperature drops due to cooling pads. & 5          & $\si{\celsius}$\\
$T^{\mathrm{sr}}$          & Reference temperature.                     & 24.5       & $\si{\celsius}$ \\

$\phi^{\mathrm{solar}}$    & Solar radiation-to-PPFD coefficient.              & 2.0                  & $\mathrm{\mu mol/(W\cdot s)}$ \\
$\phi^{\mathrm{LED}}$      & LED radiation-to-PPFD coefficient.                & 5.17                 & $\mathrm{\mu mol/(W\cdot s)}$ \\

$\lambda^{\mathrm{leak}}$  & Air leakage rate.                          & 1.0           & $\mathrm{h^{-1}}$ \\

$\zeta$                    & Empirical coefficient.                    & 0.2522   & $\mathrm{g/(m^{3} \cdot \si{\celsius})}$ \\
$\sigma$                   & Smoothing coefficient.                    & 1e-3    & $\si{\celsius}$\\

$\delta^{\mathrm{tran}}$   & Temperature sensitivity coefficient.    & 0.0518     & $\mathrm{\si{\celsius}^{-1}}$ \\
$\delta^{\mathrm{sat}}$    & Temperature sensitivity coefficient.    & 0.0572     & $\mathrm{\si{\celsius}^{-1}}$ \\
$\delta^{\mathrm{vc}}$     & Temperature sensitivity coefficient.    & 0.0485     & $\mathrm{\si{\celsius}^{-1}}$ \\
$\delta^{\mathrm{sr}}$     & Temperature sensitivity coefficient.    & 0.023      & $\mathrm{\si{\celsius}^{-2}}$ \\

$\eta^{\mathrm{LED-r}}$ & Fraction of LED electrical power emitted as radiation (light).            & 0.59 & -- \\
$\eta^{\mathrm{LED-cano}}$ & Attenuation of LED radiation to canopy.     & 0.40 & - \\
$\eta^{\mathrm{evap}}$     & Water evaporation fraction.                 & 0.70 & - \\
$\eta^{\mathrm{cover}}$    & Cover shortwave transmissivity.             & 0.50 & - \\
$\eta^{\mathrm{short}}$    & Shortwave absorption ratio.                 & 0.86 & - \\
$\eta^{\mathrm{ext}}$      & Light extinction coefficient.               & 0.70 & - \\

$\eta^{\mathrm{shad}}$     & Radiation attenuation of shading screen.            & 0.35 & - \\
$\eta^{\mathrm{warm}}$     & Heat insulation of warm-keeping screen.             & 0.50 & - \\

$\chi$                     & Latent-to-sensible ratio (sat air).         & 2.5 & -\\

$\omega^{\mathrm{Q}}$      & Thermal inertia coefficient.                & 30 & - \\
$\omega^{\mathrm{F}}$      & Humidity buffering coefficient.             & 15 & - \\

\bottomrule       
\end{tabularx}
\end{table}

The indoor temperature dynamics are modeled by \eqref{Q}.
\begin{subequations}\label{Q}
\begin{align}
&\frac{\mathrm{d}T^{\mathrm{in}}}{\mathrm{d}t} = \frac{Q^{\mathrm{heat}} - Q^{\mathrm{vent}} - Q^{\mathrm{cool}} + Q^{\mathrm{solar}} - Q^{\Delta T} + Q^{\mathrm{LED}}  - Q^{\mathrm{tran}}}{\omega^{\mathrm{Q}} \cdot V^{\mathrm{gh}} \cdot c^{\mathrm{vh}}}                                                                             \label{Q:Tdyn}\\
& Q^{\mathrm{heat}} = \bar{Q}^{\mathrm{heat}} \cdot U^{\mathrm{heat}}                  \label{Q:heat}\\
& Q^{\mathrm{vent}} = c^{\mathrm{vh}} \cdot \Delta T \cdot V^{\mathrm{vent}}\label{Q:Qvent}\\
& Q^{\mathrm{cool}} = q^{\mathrm{evap}} \cdot \left( F^{\mathrm{pad}} +\eta^{\mathrm{evap}} \cdot \bar{F}^{\mathrm{hum}} \cdot U^{\mathrm{hum}}\right) \label{Q:cool}\\
& Q^{\mathrm{solar}} = R^{\mathrm{cano-solar}} \cdot A^{\mathrm{floor}}   \label{Q:Qsol}\\
& Q^{\Delta T} = \left[
\upsilon^{\mathrm{roof}} \cdot A^{\mathrm{roof}} \cdot
\left(1-\eta^{\mathrm{warm}} \cdot U^{\mathrm{scre-warm}}\right)
+ \upsilon^{\mathrm{wall}} \cdot  A^{\mathrm{wall}}
\right] \cdot \Delta T                                                             \label{Q:QDelta}\\
& Q^{\mathrm{LED}}  = \bar{P}^{\mathrm{LED}} \cdot U^{\mathrm{LED}}                   \label{Q:QLED}\\
& Q^{\mathrm{tran}} = q^{\mathrm{evap}} \cdot g^{\mathrm{tran}} \cdot \left(H^{\mathrm{cano}} - H^{\mathrm{in}}\right) \cdot A^{\mathrm{floor}} \label{Q:Qtrans}
\end{align}
\end{subequations}

The indoor humidity dynamics are modeled by \eqref{H}.
\begin{subequations}\label{H}
\begin{align}
& \frac{\mathrm{d}H^{\mathrm{in}}}{\mathrm{d}t}
=\frac{F^{\mathrm{cool}} - F^{\mathrm{dehum}} - F^{\mathrm{vent}} + F^{\mathrm{tran}}-F^{\mathrm{vc}}}{\omega^{\mathrm{F}} \cdot V^{\mathrm{gh}}}                        \label{H:Hdyn}\\
& F^{\mathrm{cool}} = F^{\mathrm{pad}} + \eta^{\mathrm{evap}} \cdot \bar{F}^{\mathrm{hum}} \cdot U^{\mathrm{hum}}                              \label{H:Hcool}\\
& F^{\mathrm{pad}}= \frac{c^{\mathrm{vh}}\cdot \bar{V}^{\mathrm{fan}}\cdot U^{\mathrm{fan}}\cdot \Delta T^{\mathrm{pad}}\cdot U^{\mathrm{pad}}}{q^{\mathrm{evap}}}                                                 
                                                                                                                                         \label{H:Hpad}\\
& F^{\mathrm{dehum}}  = \bar{F}^{\mathrm{dehum}}\cdot U^{\mathrm{dehum}}                                                                       \label{H:Hdehum}\\
& F^{\mathrm{vent}} = (H^{\mathrm{in}} - H^{\mathrm{out}}) \cdot \left(
\bar{V}^{\mathrm{fan}} \cdot U^{\mathrm{fan}}+
\bar{V}^{\mathrm{nat}} \cdot U^{\mathrm{nat}} +
\frac{\lambda^{\mathrm{leak}} \cdot V^{\mathrm{gh}}}{3600}\right)                                                                              \label{H:Hvent}\\
& F^{\mathrm{tran}} = g^{\mathrm{tran}}\cdot \left(H^{\mathrm{cano}} - H^{\mathrm{in}}\right) \cdot A^{\mathrm{floor}}                         \label{H:Htrans}\\
& F^{\mathrm{vc}}   = g^{\mathrm{vc}}\cdot \left[\zeta \cdot \exp \left(\delta^{\mathrm{vc}} \cdot T^{\mathrm{in}} \right)\cdot\Delta T - (H^{\mathrm{in,sat}} - H^{\mathrm{in}}) \right] \cdot A^{\mathrm{floor}}                                                                                              \label{H:Hvc}
\end{align}
\end{subequations}

The indoor CO$_{2}$ concentration dynamics are modeled by \eqref{C}.
\begin{subequations}\label{C}
\begin{align}
&\frac{\mathrm{d}C^{\mathrm{in}}}{\mathrm{d}t}
=\frac{D^{\mathrm{dos}} - D^{\mathrm{vent}} - D^{\mathrm{ass}}}{V^{\mathrm{gh}}}                                 \label{C:Cdyn}\\
& D^{\mathrm{dos}}  = \bar{D}^{\mathrm{dos}} \cdot U^{\mathrm{dos}}                                              \label{C:Cdos}\\
& D^{\mathrm{vent}} = (C^{\mathrm{in}} - C^{\mathrm{out}}) \cdot
\left(
\bar{V}^{\mathrm{fan}} \cdot U^{\mathrm{fan}}+
\bar{V}^{\mathrm{nat}} \cdot U^{\mathrm{nat}} +
\frac{\lambda^{\mathrm{leak}} \cdot V^{\mathrm{gh}}}{3600}\right) \label{C:Cvent}\\
& D^{\mathrm{ass}}  = \bar{D}^{\mathrm{ass}} \cdot \frac{C^{\mathrm{in}}}{C^{\mathrm{in}} + \hat{C}^{\mathrm{in}}}\cdot \left[1 - \exp\left(- \rho \cdot X^{\mathrm{cano}}\right)\right] \cdot A^{\mathrm{floor}}   \label{C:Cass}
\end{align}
\end{subequations}

The daily light integral is modeled by \eqref{DLI}:
\begin{equation}\label{DLI}
\frac{\mathrm{d}L^{\mathrm{DLI}}}{\mathrm{d}t} =
\frac{X^{\mathrm{cano}}}{10^{6}}
\end{equation}

The intermediate variables are calculated as follows. The transpiration model, \eqref{Q:Qtrans} and \eqref{H:Htrans}, is adopted from \cite{PJM_Minimal_heating_and_cooling_in_a_modern_rose_greenhouse, Stanghellini_A_Model_of_Humidity_1995}; the condensation vapor humidity model \eqref{H:Hvc} is adopted from \cite{Bontsema_Jan_Online_estimation_of_the_transpiration, Stanghellini_A_Model_of_Humidity_1995}. The CO$_{2}$ assimilation model \eqref{C:Cass} is adopted from \cite{PJM_Optimal_control_of_greenhouse, Stanghellini_ISHS_2008, Stanghellini_ISHS_2012}.

\begin{align}
& V^{\mathrm{vent}} =
\bar{V}^{\mathrm{fan}} \cdot U^{\mathrm{fan}}+
\bar{V}^{\mathrm{nat}} \cdot U^{\mathrm{nat}} +
\frac{\lambda^{\mathrm{leak}} \cdot V^{\mathrm{gh}}}{3600} \\
& \Delta T = T^{\mathrm{in}} - T^{\mathrm{out}}\\
&g^{\mathrm{tran}} = \frac{2 \cdot L}{\left[1 + 0.7584 \cdot \exp \left(\delta^{\mathrm{tran}} \cdot T^{\mathrm{in}}\right)\right] \cdot r^{\mathrm{b}} + r^{\mathrm{s}}}\\
&r^{\mathrm{s}} = \left[\underline{r}^{\mathrm{s}} + \bar{r}^{\mathrm{s}} \cdot \exp \left(-\frac{\tau \cdot R^{\mathrm{cano-abs}}}{L} \right)\right] \cdot \left[1 + \delta^{\mathrm{sr}} \cdot(T^{\mathrm{in}} - T^{\mathrm{sr}})^{2}\right]\\
&R^{\mathrm{cano-abs}} = \eta^{\mathrm{short}} \cdot \left[\frac{\exp \left(\eta^{\mathrm{ext}} \cdot L\right) - 1}{\exp \left(\eta^{\mathrm{ext}} \cdot L\right)}\right] \cdot R^{\mathrm{cano-glo}}\\
&R^{\mathrm{cano-glo}} = R^{\mathrm{cano-solar}} + R^{\mathrm{cano-LED}}\\
&R^{\mathrm{cano-solar}} = \eta^{\mathrm{cover}} \cdot \left(1-\eta^{\mathrm{shad}} \cdot U^{\mathrm{scre-shad}}\right) \cdot R^{\mathrm{out}}\\
&R^{\mathrm{cano-LED}} = \frac{ \eta^{\mathrm{LED-r}} \cdot \eta^{\mathrm{LED-cano}} \cdot \bar{P}^{\mathrm{LED}} \cdot U^{\mathrm{LED}}}{A^{\mathrm{floor}}}               
\end{align}
from \cite{Bontsema_Jan_Online_estimation_of_the_transpiration, PJM_Minimal_heating_and_cooling_in_a_modern_rose_greenhouse, Stanghellini_A_Model_of_Humidity_1995, Goudriaan_A_Mathematical_Function_for_Crop_Growth}.

\begin{equation}
H^{\mathrm{cano}} = H^{\mathrm{in,sat}}+ \chi \cdot \frac{r^{\mathrm{b}} \cdot R^{\mathrm{cano-abs}}}{2\cdot L \cdot q^{\mathrm{evap}}}
\end{equation}
from \cite{Stanghellini_A_Model_of_Humidity_1995}.

\begin{equation}
H^{\mathrm{in,sat}}  = \tilde{H} \cdot \exp \left(\delta^{\mathrm{sat}} \cdot T^{\mathrm{in}}\right)
\end{equation}
from \cite{Bontsema_Jan_Online_estimation_of_the_transpiration, PJM_Minimal_heating_and_cooling_in_a_modern_rose_greenhouse}.


\begin{align}
&g^{\mathrm{vc}} = \nu^{\mathrm{surf}} \cdot
\left[
\frac{\left(T^{\mathrm{in}} - T^{\mathrm{cover}} \right) + \sqrt{\left(T^{\mathrm{in}} - T^{\mathrm{cover}}\right)^{2} + \sigma^{2}}}
{2} \right]^{1/3}\\
&T^{\mathrm{cover}} = \frac{2 \cdot T^{\mathrm{out}} + T^{\mathrm{in}}}{3}
\end{align}
from \cite{Bontsema_Jan_Online_estimation_of_the_transpiration, Stanghellini_A_Model_of_Humidity_1995}.

To calculate the photon flux density of photosynthetically active radiation, known as photosynthetic photon flux density (PPFD), we follow \cite{Stanghellini_ISHS_2012} and have:
\begin{equation}
X^{\mathrm{cano}}=\phi^{\mathrm{solar}} \cdot R^{\mathrm{cano-solar}} + \phi^{\mathrm{LED}} \cdot R^{\mathrm{cano-LED}}
\end{equation}

The nonlinear continuous-time dynamics are compactly written as
\begin{equation}\label{state_space_con}
\dot{\boldsymbol{x}}(t)=f\!\left(\boldsymbol{x}(t),\boldsymbol{u}(t),\boldsymbol{d}(t)\right),
\end{equation}
where $f(\cdot)$ is defined by the temperature, humidity, CO$_2$, and DLI dynamics in \eqref{Q}, \eqref{H}, \eqref{C}, and \eqref{DLI}.

\begin{table}[t]
\scriptsize
\centering
\caption{Variables to build the MPC framework.}
\renewcommand{\arraystretch}{1.20} % <-- increase vertical space
\begin{tabularx}{\linewidth}{@{}lYl@{}}
\toprule
Notation             & Definition                              & Unit \\
\midrule
$B^{\mathrm{in}}_{k}$& Indoor CO$_2$ mixing ratio.                                                                                           & $\mathrm{ppm}$ \\
$B^{\star}_{k}$      & Reference for $B^{\mathrm{in}}_{k}$, $B^{\star}_{k} = (\bar{B}^{\mathrm{in}}_{k}+\underline{B}^{\mathrm{in}}_{k})/2$. & $\mathrm{ppm}$\\


$E^{\mathrm{heat}}_{k}$  & Cost of heating in step $k$.            & \$ \\
$E^{\mathrm{fan}}_{k}$   & Cost of fan electricity in step $k$.    & \$ \\
$E^{\mathrm{LED}}_{k}$   & Cost of LED electricity in step $k$.    & \$ \\
$E^{\mathrm{pad}}_{k}$   & Cost of cooling pad in step $k$.        & \$ \\
$E^{\mathrm{hum}}_{k}$   & Cost of humidifying in step $k$.        & \$ \\
$E^{\mathrm{dehum}}_{k}$ & Cost of dehumidifying in step $k$.      & \$ \\
$E^{\mathrm{dos}}_{k}$   & Cost of CO$_2$ dosing in step $k$.      & \$ \\

$k$                  & Step index.                             & -- \\
$\kappa_k$           & Time-of-day step index, $\kappa_k \in \{0,\dots,287\}$. & -- \\

$L^{\mathrm{DLI}\star}_{k}$ & Reference for $L^{\mathrm{DLI}}_{k}$.
$\begin{cases}
0, & 0 \le \kappa_k < 72,\\
22\cdot\frac{\kappa_k-72}{192}, & 72 \le \kappa_k < 264,\\
22, & 264 \le \kappa_k \le 287,
\end{cases}$                                                                                                                         & $\mathrm{mol/m^{2}}$ \\
$L^{\Delta}_{k}$            & DLI shortage, $L^{\Delta}_{k} = \max (0,L^{\mathrm{DLI}\star}_{k}-L^{\mathrm{DLI}}_{k})$.   & $\mathrm{mol/m^{2}}$ \\

$S_{k}$            & Slacking variables.   & Depends on slack terms \\


$T^{\star}_{k}$ & Reference for $T^{\mathrm{in}}_{k}$, $T^{\star}_{k} = (\bar{T}^{\mathrm{in}}_{k}+\underline{T}^{\mathrm{in}}_{k})/2$. & $\si{\celsius}$ \\

$Z^{\mathrm{in}}_{k}$ & Indoor vapor pressure deficit.                                                                                          & $\mathrm{kPa}$\\
$Z^{\star}_{k}$       & Reference for $Z^{\mathrm{in}}_{k}$, $Z^{\star}_{k} = (\bar{Z}^{\mathrm{in}}_{k}+\underline{Z}^{\mathrm{in}}_{k})/2$. & $\mathrm{kPa}$\\
\bottomrule 
\end{tabularx}
\end{table}

\begin{table}[t]
\scriptsize
\centering
\caption{Parameters to build the MPC framework (values shown as day/night when applicable).}
\renewcommand{\arraystretch}{1.20}
\begin{tabularx}{\linewidth}{@{}lYll@{}}
\toprule
Notation & Definition & Value & Unit \\
\midrule
$\alpha^{\mathrm{heat}}$  & Price of heating.       & 0.05     & $\mathrm{\$/kWh}$ \\
$\alpha^{\mathrm{elec}}$  & Price of electricity.   & 0.125    & $\mathrm{\$/kWh}$ \\
$\alpha^{\mathrm{water}}$ & Price of water.         & 1.48e-6  & $\mathrm{\$/g}$\\
$\alpha^{\mathrm{hum}}$   & Price of humidifying.   & 4.79e-6  & $\mathrm{\$/g}$\\
$\alpha^{\mathrm{dehum}}$ & Price of dehumidifying. & 6.5e-5   & $\mathrm{\$/g}$\\
$\alpha^{\mathrm{dos}}$   & Price of CO$_2$ dosing. & 1.5e-4   & $\mathrm{\$/g}$\\

$\beta^{\mathrm{C}}$ & CO$_2$ unit conversion factor. & 549.2 & $\mathrm{ppm\cdot m^{3}/g}$ \\

$\tilde{H}^{\mathrm{sat}}$ & Saturation absolute humidity at 22$\si{\celsius}$. & 19.4 & $\mathrm{g/m^{3}}$\\


$K$        & Number of total steps. & 12   & --\\
$S^{\mathrm{fan}}$ & Specific fan power. & 93.4 & $\mathrm{W/(m^{3}/s)}$ \\
$\Delta t$ & Time interval.          & 300  & $\mathrm{s}$ \\
$\gamma$   & Discount factor.        & 0.99 & --\\

$\bar{T}^{\mathrm{in}}_{k}$       & Upper bound of $T^{\mathrm{in}}_{k}$. & 25/19     & $\si{\celsius}$ \\
$\underline{T}^{\mathrm{in}}_{k}$ & Lower bound of $T^{\mathrm{in}}_{k}$. & 23/18     & $\si{\celsius}$ \\

$\bar{Z}^{\mathrm{in}}_{k}$       & Upper bound of $Z^{\mathrm{in}}_{k}$. & 1.30      & $\mathrm{kPa}$\\
$\underline{Z}^{\mathrm{in}}_{k}$ & Lower bound of $Z^{\mathrm{in}}_{k}$. & 0.53      & $\mathrm{kPa}$\\

$\bar{B}^{\mathrm{in}}_{k}$       & Upper bound of $B^{\mathrm{in}}_{k}$. & 1050/2000 & $\mathrm{ppm}$\\
$\underline{B}^{\mathrm{in}}_{k}$ & Lower bound of $B^{\mathrm{in}}_{k}$. & 950/0     & $\mathrm{ppm}$\\

$\lambda^{\mathrm{T}}$     & Tracking coefficient for $T^{\mathrm{in}}_{k}$. & 0.20   & $\mathrm{\$/\si{\celsius}^2}$\\
$\lambda^{\mathrm{Z}}$     & Tracking coefficient for $Z^{\mathrm{in}}_{k}$. & 0.05   & $\mathrm{\$/kPa^2}$\\
$\lambda^{\mathrm{B}}_{k}$ & Tracking coefficient for $B^{\mathrm{in}}_{k}$. & 0.03/0 & $\mathrm{\$/ppm^2}$\\
$\lambda^{\mathrm{L}}$     & Tracking coefficient for $L^{\mathrm{DLI}}_{k}$.& 0.10   & $\mathrm{\$/(mol^2/m^4)}$\\

$\xi$ & Scaled vapor gas constant at 22$\si{\celsius}$. & 0.1362 & $\mathrm{kPa\cdot m^{3}/g}$\\
\bottomrule
\end{tabularx}
\end{table}

Next, we define more variables and parameters to build the MPC framework.

State vector: $\boldsymbol{x}(t) = 
\left[
T^{\mathrm{in}},
H^{\mathrm{in}},
C^{\mathrm{in}},
L^{\mathrm{DLI}}
\right]^{\top}$

Initial state vector: $\boldsymbol{x}^{\mathrm{ini}} = \left[
19   \,\si{\celsius},
7.41 \,\mathrm{g/m^{3}},
0.92 \,\mathrm{g/m^{3}},
0    \,\mathrm{\mathrm{mol/m^{2}}}
\right]^{\top}$

Control input vector:

$\boldsymbol{u}(t) =
\left[
U^{\mathrm{heat}},
U^{\mathrm{fan}},
U^{\mathrm{nat}},
U^{\mathrm{pad}},
U^{\mathrm{dos}},
U^{\mathrm{LED}},
U^{\mathrm{hum}},
U^{\mathrm{dehum}},
U^{\mathrm{scre-shad}},
U^{\mathrm{scre-warm}}
\right]^{\top}
$

For an MPC framework with a time step of length $\Delta t$, the state-space model can be written in a discrete form using Euler integration \eqref{statespace_d}.
\begin{equation}\label{statespace_d}
\boldsymbol{x}_{k+1} = \boldsymbol{x}_{k} + \Delta t\cdot f(\boldsymbol{x}_{k},\boldsymbol{u}_{k},\boldsymbol{d}_{k})
\end{equation}

Outdoor disturbance vector:
$
\boldsymbol{d}(t) = 
\left[
T^{\mathrm{out}},
H^{\mathrm{out}},
C^{\mathrm{out}},
R^{\mathrm{out}}
\right]^{\top}
$

Cost of resource consumption of the greenhouse during step $k$:
\begin{align}
& E^{\mathrm{heat}}_{k} =\alpha^{\mathrm{heat}}\cdot Q^{\mathrm{heat}}_{k} \cdot \frac{\Delta t}{3.6 \times 10^{6}}\\
& E^{\mathrm{fan}}_{k}  =\alpha^{\mathrm{elec}}\cdot S^{\mathrm{fan}}\cdot\bar{V}^{\mathrm{fan}}\cdot U^{\mathrm{fan}}_{k}\cdot\frac{\Delta t}{3.6\times 10^{6}}\\
& E^{\mathrm{LED}}_{k}  =\alpha^{\mathrm{elec}} \cdot \bar{P}^{\mathrm{LED}}   \cdot U^{\mathrm{LED}}_{k} \cdot \frac{\Delta t}{3.6\times 10^{6}}\\
& E^{\mathrm{pad}}_{k}  =\alpha^{\mathrm{water}}\cdot U^{\mathrm{pad}}_{k} \cdot \Delta t\\
& E^{\mathrm{hum}}_{k}  =\alpha^{\mathrm{hum}}  \cdot \bar{F}^{\mathrm{hum}}   \cdot U^{\mathrm{hum}}_{k} \cdot \Delta t \\
& E^{\mathrm{dehum}}_{k}=\alpha^{\mathrm{dehum}}\cdot \bar{F}^{\mathrm{dehum}} \cdot U^{\mathrm{dehum}}_{k} \cdot \Delta t\\
& E^{\mathrm{dos}}_{k}  =\alpha^{\mathrm{dos}}  \cdot \bar{D}^{\mathrm{dos}}   \cdot U^{\mathrm{dos}}_{k} \cdot \Delta t
\end{align}

The discrete nonlinear MPC model is given in \eqref{MPC_discrete_nonlinear}.
\begin{subequations}\label{MPC_discrete_nonlinear}
\begin{flalign}
\min_{\boldsymbol{u}_{0},...,\boldsymbol{u}_{K-1}}
& \sum_{k=0}^{K-1} \gamma^{k} \left[
\begin{aligned}
&  E^{\mathrm{heat}}_{k}
 + E^{\mathrm{fan}}_{k}
 + E^{\mathrm{LED}}_{k}
 + E^{\mathrm{pad}}_{k}
 + E^{\mathrm{hum}}_{k}
 + E^{\mathrm{dehum}}_{k}
 + E^{\mathrm{dos}}_{k} \\
& + \lambda^{\mathrm{T}} \left(T^{\mathrm{in}}_{k}  - T^{\star}_{k}\right)^{2}
  + \lambda^{\mathrm{Z}} \left(Z^{\mathrm{in}}_{k}  - Z^{\star}_{k}\right)^{2}
  + \lambda^{\mathrm{B}}_{k}\left(B^{\mathrm{in}}_{k}  - B^{\star}_{k}\right)^{2}
  + \lambda^{\mathrm{L}} \left(L^{\Delta}_{k}\right)^{2}\\
& + \lambda^{\mathrm{T+}} S^{\mathrm{T+}}_{k} + \lambda^{\mathrm{T-}} S^{\mathrm{T-}}_{k}
  + \lambda^{\mathrm{Z+}} S^{\mathrm{Z+}}_{k} + \lambda^{\mathrm{Z-}} S^{\mathrm{Z-}}_{k}
  + \lambda^{\mathrm{B+}} S^{\mathrm{B+}}_{k} + \lambda^{\mathrm{B-}} S^{\mathrm{B-}}_{k}
\end{aligned}
\right]
\mspace{-400mu} &\notag\\
\text{s.t.}\quad
& \boldsymbol{x}_{0} = \boldsymbol{x}^{\mathrm{ini}},
\mspace{-400mu} & \\
& \boldsymbol{x}_{k+1} = \boldsymbol{x}_{k} + \Delta t\cdot f(\boldsymbol{x}_{k},\boldsymbol{u}_{k},\boldsymbol{d}_{k}),
\mspace{-400mu} & k = 0,...,K-1,\\
& 0 \le \boldsymbol{u}_{k} \le 1,
\mspace{-400mu} & k = 0,...,K-1, \\
& Z^{\mathrm{in}}_{k}= \left(\tilde{H}^{\mathrm{sat}} - H^{\mathrm{in}}_{k}\right) \cdot \xi,
\mspace{-400mu} & k = 0,...,K-1, \\
& B^{\mathrm{in}}_{k} = \beta^{\mathrm{C}} \cdot C^{\mathrm{in}}_{k},  \mspace{-200mu} & k = 0,...,K-1, \\
&\underline{T}^{\mathrm{in}}_{k} + S^{\mathrm{T-}}_{k} \le T^{\mathrm{in}}_{k}\le \bar{T}^{\mathrm{in}}_{k} + S^{\mathrm{T+}}_{k},
\mspace{-400mu} & k = 0,...,K-1, \\
&\underline{Z}^{\mathrm{in}}_{k} + S^{\mathrm{Z-}}_{k} \le Z^{\mathrm{in}}_{k}\le \bar{Z}^{\mathrm{in}}_{k} + S^{\mathrm{Z+}}_{k},
\mspace{-400mu} & k = 0,...,K-1, \\
&\underline{B}^{\mathrm{in}}_{k} + S^{\mathrm{B-}}_{k} \le B^{\mathrm{in}}_{k}\le \bar{B}^{\mathrm{in}}_{k} + S^{\mathrm{B+}}_{k},
\mspace{-400mu} & k = 0,...,K-1, \\
& L^{\Delta}_{k} \ge L^{\mathrm{DLI}\star}_{k} - L^{\mathrm{DLI}}_{k},
\mspace{-400mu} & k = 0,...,K-1, \\
& L^{\Delta}_{k} \ge 0,
\mspace{-400mu} & k = 0,...,K-1,\\
& U^{\mathrm{fan}}_{k} \geq U^{\mathrm{pad}}_{k},
\mspace{-400mu} & k = 0,...,K-1,\\
& S^{\mathrm{T-}}_{k}, S^{\mathrm{T+}}_{k},  S^{\mathrm{Z-}}_{k}, S^{\mathrm{Z+}}_{k}, S^{\mathrm{B-}}_{k}, S^{\mathrm{B+}}_{k} \geq 0,
\mspace{-400mu} & k = 0,...,K-1.
\end{flalign}
\end{subequations}


The linear MPC model is given in \eqref{MPC}.
\begin{subequations}\label{MPC}
\begin{flalign}
\min_{\boldsymbol{u}_{0},...,\boldsymbol{u}_{K-1}}
& \sum_{k=0}^{K-1} \gamma^{k} \left[
\begin{aligned}
&  E^{\mathrm{heat}}_{k}
 + E^{\mathrm{fan}}_{k}
 + E^{\mathrm{LED}}_{k}
 + E^{\mathrm{pad}}_{k}
 + E^{\mathrm{hum}}_{k}
 + E^{\mathrm{dehum}}_{k}
 + E^{\mathrm{dos}}_{k} \\
& + \lambda^{\mathrm{T}} \left(T^{\mathrm{in}}_{k}  - T^{\star}_{k}\right)^{2}
  + \lambda^{\mathrm{Z}} \left(Z^{\mathrm{in}}_{k}  - Z^{\star}_{k}\right)^{2}
  + \lambda^{\mathrm{B}}_{k}\left(B^{\mathrm{in}}_{k}  - B^{\star}_{k}\right)^{2}
  + \lambda^{\mathrm{L}} \left(L^{\Delta}_{k}\right)^{2}\\
& + \lambda^{\mathrm{T+}} S^{\mathrm{T+}}_{k} + \lambda^{\mathrm{T-}} S^{\mathrm{T-}}_{k}
  + \lambda^{\mathrm{Z+}} S^{\mathrm{Z+}}_{k} + \lambda^{\mathrm{Z-}} S^{\mathrm{Z-}}_{k}
  + \lambda^{\mathrm{B+}} S^{\mathrm{B+}}_{k} + \lambda^{\mathrm{B-}} S^{\mathrm{B-}}_{k}
\end{aligned}
\right]                                                                                  \mspace{-400mu} &\notag\\
\text{s.t.}\quad
& \boldsymbol{x}_{0} = \boldsymbol{x}^{\mathrm{ini}},                                    \mspace{-400mu} & \\
& \boldsymbol{x}_{k+1} = \boldsymbol{M} \boldsymbol{x}_k + \boldsymbol{N} \boldsymbol{u}_k + \boldsymbol{O} \boldsymbol{d}_k + \boldsymbol{m},
                                                                                         \mspace{-400mu} & k = 0,...,K-1, \\
& 0 \le \boldsymbol{u}_{k} \le 1,
\mspace{-400mu} & k = 0,...,K-1, \\
& Z^{\mathrm{in}}_{k}= \left(\tilde{H}^{\mathrm{sat}} - H^{\mathrm{in}}_{k}\right) \cdot \xi,
\mspace{-400mu} & k = 0,...,K-1, \\
& B^{\mathrm{in}}_{k} = \beta^{\mathrm{C}} \cdot C^{\mathrm{in}}_{k},  \mspace{-200mu} & k = 0,...,K-1, \\
&\underline{T}^{\mathrm{in}}_{k} + S^{\mathrm{T-}}_{k} \le T^{\mathrm{in}}_{k}\le \bar{T}^{\mathrm{in}}_{k} + S^{\mathrm{T+}}_{k},
\mspace{-400mu} & k = 0,...,K-1, \\
&\underline{Z}^{\mathrm{in}}_{k} + S^{\mathrm{Z-}}_{k} \le Z^{\mathrm{in}}_{k}\le \bar{Z}^{\mathrm{in}}_{k} + S^{\mathrm{Z+}}_{k},
\mspace{-400mu} & k = 0,...,K-1, \\
&\underline{B}^{\mathrm{in}}_{k} + S^{\mathrm{B-}}_{k} \le B^{\mathrm{in}}_{k}\le \bar{B}^{\mathrm{in}}_{k} + S^{\mathrm{B+}}_{k},
\mspace{-400mu} & k = 0,...,K-1, \\
& L^{\Delta}_{k} \ge L^{\mathrm{DLI}\star}_{k} - L^{\mathrm{DLI}}_{k},
\mspace{-400mu} & k = 0,...,K-1, \\
& L^{\Delta}_{k} \ge 0,
\mspace{-400mu} & k = 0,...,K-1,\\
& U^{\mathrm{fan}}_{k} \geq U^{\mathrm{pad}}_{k},
\mspace{-400mu} & k = 0,...,K-1,\\
& S^{\mathrm{T-}}_{k}, S^{\mathrm{T+}}_{k},  S^{\mathrm{Z-}}_{k}, S^{\mathrm{Z+}}_{k}, S^{\mathrm{B-}}_{k}, S^{\mathrm{B+}}_{k} \geq 0,
\mspace{-400mu} & k = 0,...,K-1.
\end{flalign}
\end{subequations}
where $\Delta \boldsymbol{x}_{k}$ is the state incremental term; $\boldsymbol{M}$, $\boldsymbol{N}$, and $\boldsymbol{O}$ are the discrete-time system matrices, and $\boldsymbol{m}$ is the constant offset term.



\bibliographystyle{elsarticle-num} 
\bibliography{cas-refs}

\end{document}